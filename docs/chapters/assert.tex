
\chapter{Assertion Testing}

\section{Introduction}
This module is used for writing unit tests for your applications, you can access it with \texttt{require('assert')}.\section{assert.fail}
\subsection*{Syntax:}
\begin{center}\texttt{assert.fail(actual, expected, message, operator)}

\end{center}where
\begin{itemize}
\item \texttt{actual}: the actual value returned by your function
\item \texttt{expected}: the value you expect your function returns
\item \texttt{message}: the message you want to display when an exception is thrown
\item \texttt{operator}: the operator is used to separate the values for \texttt{actual} and \texttt{expected} when \texttt{message} is falsy.
\end{itemize}
This function throws an exception that displays the message your provided via parameter \texttt{message} or the values for \texttt{actual} and \texttt{expected} separated by the delimiter you provide via parameter \texttt{separator}.

\subsection{Three Arguments}
Let's start with an example with three arguments: \texttt{actual}, \texttt{expected} and \texttt{message}, which is also used most often in practice. It is shown in \textit{Listing~\ref{001_fail_with_message.js}}.
\lstinputlisting[caption={\textit{An exception is thrown with customized message}},label={001_fail_with_message.js}]{../src/assert/001_fail_with_message.js}
Running the code in \textit{Listing~\ref{001_fail_with_message.js}} produces the log as shown in \textit{Listing~\ref{001_fail_with_message.log}}.\lstinputlisting[caption={\textit{The error messages of running script in Listing~\ref{001_fail_with_message.js}}},label={001_fail_with_message.log}]{../src/assert/001_fail_with_message.log}
A few takeaways from the error messages in \textit{Listing~\ref{001_fail_with_message.log}}.
\begin{itemize}
\item the exception thrown by \texttt{assert.fail} is an instance of \texttt{assert.AssertionError}, which inherits from Javascript's \texttt{Error} object.
\item the error message \texttt{"1 is not equal to 2"} is the customized message provided by you.
\end{itemize}
We can also put the function call into a \texttt{try...catch} statement as shown in \textit{Listing~\ref{001_fail_with_message_catch.js}}.
\lstinputlisting[caption={\textit{An exception is thrown and caught}},label={001_fail_with_message_catch.js}]{../src/assert/001_fail_with_message_catch.js}
\subsection{Four Arguments}
The fourth argument \texttt{operator} will be used to separate the the actual value and expected value when the third argument \texttt{message} is falsy. We add the fourth argument as shown in \textit{Listing~\ref{001_fail_truthy_message.js}}.
\lstinputlisting[caption={\textit{An exception is thrown with truthy message}},label={001_fail_truthy_message.js}]{../src/assert/001_fail_truthy_message.js}
Running the code in \textit{Listing~\ref{001_fail_truthy_message.js}} produces the error stack trace as shown in \textit{Listing~\ref{001_fail_truthy_message.log}}.\lstinputlisting[caption={\textit{The error messages of running script in Listing~\ref{001_fail_truthy_message.js}}},label={001_fail_truthy_message.log}]{../src/assert/001_fail_truthy_message.log}
An astute reader will notice that the error message in \textit{Listing~\ref{001_fail_truthy_message.log}} is the same as the one in \textit{Listing~\ref{001_fail_with_message.log}}, which indicates that the fourth argument \texttt{operator} is not used when the third argument \texttt{message} is truthy. Let's see what will happen if the third argument is falsy. The code snippet in \textit{Listing~\ref{001_fail_empty_message.js}} shows the changes.\lstinputlisting[caption={\textit{An exception is thrown with falsy message}},label={001_fail_empty_message.js}]{../src/assert/001_fail_empty_message.js}
The error message in \textit{Listing~\ref{001_fail_truthy_message.log}} shows that the the \texttt{operator} we provided is used to separate the actual value and the expected value.\lstinputlisting[caption={\textit{The error messages of running script in Listing~\ref{001_fail_empty_message.js}}},label={001_fail_empty_message.log}]{../src/assert/001_fail_empty_message.log}
\section{assert.ok}
\label{sec:assert.ok}
\subsection*{Syntax:}
\begin{center}\texttt{assert.ok(value, [message])}

\end{center}where
\begin{itemize}
\item \texttt{value}: check whether the value is truthy
\item \texttt{message}: the message you want to display when an exception is thrown
\end{itemize}

This function is equivalent to function \texttt{assert.equal(!!value, true, message);} in Section~\ref{sec:assert.equal}.
\subsection{One Argument}
The script in \textit{Listing~\ref{002_ok_empty_message.js}} shows two simple examples of using this function. The error message in \textit{Listing~\ref{002_ok_empty_message.log}} shows operator "\texttt{==}" is used to separate the actual value and the expected value.
\lstinputlisting[caption={\textit{One is Okay and the other is not Okay}},label={002_ok_empty_message.js}]{../src/assert/002_ok_empty_message.js}
\lstinputlisting[caption={\textit{The error messages of running script in Listing~\ref{002_ok_empty_message.js}}},label={002_ok_empty_message.log}]{../src/assert/002_ok_empty_message.log}
\subsection{Two Arguments}
The script in \textit{Listing~\ref{002_ok_with_message.js}} shows two examples with customized message. The customized message shows up in the error log in \textit{Listing~\ref{002_ok_empty_message.log}}.
\lstinputlisting[caption={\textit{One is Okay and the other is not Okay: with optional truthy message}},label={002_ok_with_message.js}]{../src/assert/002_ok_with_message.js}
\lstinputlisting[caption={\textit{The error messages of running script in Listing~\ref{002_ok_with_message.js}}},label={002_ok_empty_message.log}]{../src/assert/002_ok_empty_message.log}
\section{assert}
\subsection*{Syntax:}
\begin{center}\texttt{assert(value, [message])}

\end{center}where
\begin{itemize}
\item \texttt{value}: check whether the value is truthy
\item \texttt{message}: the message you want to display when an exception is thrown
\end{itemize}

This function is exactly the same as function \texttt{assert.ok}. In fact, in the source code for module \texttt{assert}, you can see the line below:

\lstinline$var assert = module.exports = ok;$
\section{assert.equal}
\label{sec:assert.equal}
\subsection*{Syntax:}
\begin{center}\texttt{assert.equal(actual, expected, [message])}

\end{center}where
\begin{itemize}
\item \texttt{actual}: the actual value returned by your function
\item \texttt{expected}: the value you expect your function returns
\item \texttt{message}: the message you want to display when an exception is thrown
\end{itemize}

The \lstinline$assert.equal$ method is implemented in Node.js as follows:
\lstinputlisting[caption={\textit{The implementation of \lstinline$assert.equal$ in Node.js}},label={src.assert.equal.js}]{../src/assert/src.assert.equal.js}
It is clear to see in \textit{Listing~\ref{src.assert.equal.js}} that \lstinline$assert.equal$ performs shallow, coercive equality test with the equal comparison operator (==). Let's verify this with a few examples.\lstinputlisting[caption={\textit{Passing examples of \lstinline$assert.equal$}},label={003_equal_passed.js}]{../src/assert/003_equal_passed.js}
All examples in \textit{Listing~\ref{003_equal_passed.js}} pass. \textit{Listing~\ref{003_equal_failed.js}} provides a few failing examples. You may wonder why the assertion in Line \ref{line_equal_failed} fails. Though the two objects have the same properties and values, they refer to two physically different objects. In another word, \lstinline$assert.equal$ checks whether 
\lstinline$actual$ and \lstinline$expected$ points two the same object when both of them are objects.\lstinputlisting[caption={\textit{Failing examples of \lstinline$assert.equal$}},label={003_equal_failed.js}]{../src/assert/003_equal_failed.js}
